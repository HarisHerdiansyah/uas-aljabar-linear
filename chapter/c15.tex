% Pertemuan 15
\section*{Pertemuan 15}
\begin{enumerate}
	% Halaman 38-39
	\item Misalkan $u=(u_1,u_2)$ dan $v=(v_1,v_2)$ adalah vektor di $\mathbb{R}^2$ dan diketahui:
	\[\langle u, v \rangle = 3u_1v_1 + 2u_2v_2\]
	Jika diketahui vektor $\mathbf{a}=(2,-3)$ dan $\mathbf{b}=(1,4)$, hitunglah nilai $\langle a, b \rangle$ dengan definisi di atas.
	
	% Halaman 40
	\item Pada ruang vektor $M_{2\times2}$ hasil kali dalam standar didefinisikan sebagai berikut:
	\[\langle A,B \rangle=tr(A^{T}B)\]
	Hitung $\langle A,B \rangle$ jika diketahui:
	\[A=\begin{bmatrix}
		1 & 2 \\ 0 & 1
	\end{bmatrix},
	B=\begin{bmatrix}
		2 & 0 \\ 1 & -1	
	\end{bmatrix}\]
	
	% Halaman 37
	\item Diketahui basis $S=\{u_1,u_2\}$ untuk ruang vektor $\mathbb{R}^2$ dengan vektor:
	\[u_1=(1,1) \text{ dan } u_2=(0,2)\]
	Gunakan proses Gram-Schmidt untuk mengubhas basis $S$ menjadi basis ortogonal.
\end{enumerate}

\textbf{Jawaban:}
\begin{enumerate}
	\item $\langle a,b \rangle$ adalah notasi untuk ruang hasil kali dalam, di mana persamaan tersebut mendefinisikan operasi kali titik (\emph{dot product}) dengan perluasan aturan tertentu.
	\[\langle \mathbf{a}, \mathbf{b} \rangle = 3(2)(1) + 2(-3)(4) = 6 + (-24) = -18\]
	\item Cari dahulu $A^T$
	\[A^T = \begin{bmatrix} 1 & 0 \\ 2 & 1 \end{bmatrix}\]
	\[A^T B = \begin{bmatrix} 1 & 0 \\ 2 & 1 \end{bmatrix} \begin{bmatrix} 2 & 0 \\ 1 & -1 \end{bmatrix} = \begin{bmatrix} (2+0) & (0+0) \\ (4+1) & (0-1) \end{bmatrix} = \begin{bmatrix} 2 & 0 \\ 5 & -1 \end{bmatrix}\]
	\[\text{tr}(A^T B) = 2 + (-1) = 1\]
	
	\item Basis ortogonal adalah basis di mana setiap vektornya saling tegak lurus. Vektor yang saling tegak lurus ini akan bernilai nol ketika dioperasikan dengan kali titik (\emph{dot product}). Adapun, basis ortonormal di mana setiap basisnya saling ortogonal dan panjang vektornya adalah 1.\\
	Gram-schmidt adalah metode yang melibatkan konsep vektor proyeksi untuk membuat sekumpulan vektor yang bebas linear menjaid basis ortogonal.
	\begin{enumerate}
		\renewcommand{\labelenumi}{\alph{enumi}.}
		\item Tetapkan $u_1$ sebagai $v_1$
		\[v_1=u_1\]
		\item Hitung $v_2$
		\[v_2=u_2-proj_{v_1}u_2\]
		\[=u_2-\frac{\langle u_2,v_1 \rangle}{||v_1||^2}v_1\]
		Hitung $\langle u_2,v_1 \rangle$:
		\[u_2 \cdot v_1=(0)(1)+(2)(1)=2\]
		Hitung $||v_1||^2$:
		\[||v_1||^2=(\sqrt{1^2+1^2})^2=(\sqrt{2})^2=2\]
		Hitung $proj_{v_1}u_2$:
		\[proj_{v_1}u_2=\frac{2}{2}\cdot(1,2)\]
		\[=(1,1)\]
		Hitung persamaan akhir untuk mendapatkan $v_2$:
		\[v_2=u_2-(1,1)\]
		\[=(0,2)-(1,1)\]
		\[=(0-1,2-1)\]
		\[=(-1,1)\]
		\item Didapat $v_1=(1,1)$ dan $v_2=(-1,1)$
	\end{enumerate}
	
	Verifikasi ortogonalitas (ketika hasil kali titik bernilai nol):
	\[\langle v_1, v_2 \rangle=(1)(-1)+(1)(1)\]
	\[=-1+1=0\]
\end{enumerate}