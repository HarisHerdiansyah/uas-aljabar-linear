% Pertemuan 1
\section*{Pertemuan 1}
Diketahui matriks $A, B, C, D, $ dan $E$ memiliki ukuran sebagai berikut: \\
\[
	A : (3 \times 6),
	B : (6 \times 3),
	C : (3 \times 5),
	D : (5 \times 6),
	E : (3 \times 2)
\]
Tentukan apakah ekspresi matriks berikut terdefinisi. Untuk yang terdefinisi berikan ukuran matriksnya hasilnya.

\begin{enumerate}
	\item $tr(DE^{T})$
	\item $tr(BC)$
\end{enumerate}

\textbf{Jawaban:}
\begin{enumerate}
	\item Matriks $DE^T$ adalah perkalian dua matriks antara matriks $D$ dan $E^T$. Ordo matriks $E$ adalah $3\times2$ dan setelah ditranspos ($E^T$) menjadi $2\times3$. Matriks $DE^T$ \textbf{tidak terdefinisi} karena syarat perkalian dua matriks adalah kolom matriks pertama harus sama dengan baris matriks kedua. Ekspresi ordo matriks $D$ dan $E$ tidak memenuhi syarat tersebut.
	
	\item Matriks BC terdefinsi karena syarat perkalian dua matriks terpenuhi, di mana kolom matriks $B$ adalah 3 dan baris matriks $C$ adalah 3 sehingga operasi perkalian dapat dikerjakan.
\end{enumerate}