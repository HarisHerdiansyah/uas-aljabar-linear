% Pertemuan 7
\section*{Pertemuan 7}
\begin{enumerate}
	\item Diketahui vektor di $\mathbb{R}^3$: $a=(-2,4,1), b=(3,0,-5), c=(1,-2,3)$ \\
	Hitung vektor hasil dari:
	\begin{enumerate}
		\renewcommand{\labelenumi}{\alph{enumi}.}
		\item $2a-5a$
		\item $2c-3(b+2a)$
	\end{enumerate}
	
	\item Diketahui vektor di $\mathbb{R}^5$: $p=(4,1,-2,3,0), q=(-1,0,5,2,-3), r=(2,-3,1,0,4)$ \\
	Tentukan hasil dari operasi berikut:
	\begin{enumerate}
		\renewcommand{\labelenumi}{\alph{enumi}.}
		\item $(4p-2q)-(3p+r)$
		\item $2(q-3r+p)-q$
	\end{enumerate}
\end{enumerate}
\textbf{Jawaban:}\\
	Operasi dasar vektor (penjumlahan, perngurangan, dan perkalian dengan skalar) konsepnya mirip seperti matriks, di mana entri-entri yang dioperasikan adalah entri yang posisinya bersesuaian. Misal diketahui $\mathbf{u}$ dan $\mathbf{v}$:
	\[\mathbf{u}=(u_1,u_2,u_3)\]
	\[\mathbf{v}=(v_1,v_2,v_3)\]
	\[\mathbf{u+v}=(u_1+v_1,u_2+v_2,u_3+v_3)\]
\begin{enumerate}	
	\item
	\begin{enumerate}
		\renewcommand{\labelenumi}{\alph{enumi}.}
		\item \begin{align*}
        		2a - 5a &= 2(-2, 4, 1) - 5(-2, 4, 1) \\
        		&= (-4, 8, 2) - (-10, 20, 5) \\
        		&= (-4 - (-10), 8 - 20, 2 - 5) \\
        		&= (6, -12, -3)
    		\end{align*}
    
    		\item \begin{align*}
        		b + 2a &= (3, 0, -5) + 2(-2, 4, 1) \\
            &= (3, 0, -5) + (-4, 8, 2) \\
            &= (3-4, 0+8, -5+2) \\
            &= (-1, 8, -3)
    		\end{align*}
    		\begin{align*}
        		2c - 3(b + 2a) &= 2(1, -2, 3) - 3(-1, 8, -3) \\
            &= (2, -4, 6) - (-3, 24, -9) \\
            &= (2 - (-3), -4 - 24, 6 - (-9)) \\
            &= (2 + 3, -28, 6 + 9) \\
            &= (5, -28, 15)
    		\end{align*}
	\end{enumerate}
	
	\item
	\begin{enumerate}
		\renewcommand{\labelenumi}{\alph{enumi}.}
		\item \begin{align*}
			(4p - 2q) - (3p + r) &= 4p - 2q - 3p - r \\
			&= (4p - 3p) - 2q - r \\
			&= p - 2q - r
		\end{align*}
		\begin{align*}
			p - 2q - r &= (4, 1, -2, 3, 0) - 2(-1, 0, 5, 2, -3) - (2, -3, 1, 0, 4) \\
			&= (4, 1, -2, 3, 0) + (2, 0, -10, -4, 6) + (-2, 3, -1, 0, -4) \\
			&= (4 + 2 - 2, 1 + 0 + 3, -2 - 10 - 1, 3 - 4 + 0, 0 + 6 - 4) \\
			&= (4, 4, -13, -1, 2)
		\end{align*}
		
		\item \begin{align*}
			2(q - 3r + p) - q &= 2q - 6r + 2p - q \\
			&= (2q - q) + 2p - 6r \\
			&= q + 2p - 6r
		\end{align*}
		\begin{align*}
			q + 2p - 6r &= (-1, 0, 5, 2, -3) + 2(4, 1, -2, 3, 0) - 6(2, -3, 1, 0, 4) \\
			&= (-1, 0, 5, 2, -3) + (8, 2, -4, 6, 0) + (-12, 18, -6, 0, -24) \\
			&= (-1 + 8 - 12, 0 + 2 + 18, 5 - 4 - 6, 2 + 6 + 0, -3 + 0 - 24) \\
			&= (-5, 20, -5, 8, -27)
		\end{align*}
	\end{enumerate}
\end{enumerate}