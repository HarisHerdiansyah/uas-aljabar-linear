% Pertemuan 12
\section*{Pertemuan 12}
\begin{enumerate}
	\item Misalkan $V$ adalah himpunan semua pasangan bilangan real $(x, y)$ di $\mathbb{R}^2$. Operasi penjumlahan didefinisikan secara standar, yaitu:
	\[(x_1, y_1) + (x_2, y_2) = (x_1+x_2, \ y_1+y_2)\]
	Namun, operasi perkalian skalar didefinsikan sebagai berikut:
	\[k(x, y) = (kx, 0)\]
	Tentukan apakah $V$ merupakan ruang vektor atau bukan
	
	\item Misalkan $W$ adalah himpunan semua matriks $2\times2$ di mana $tr(M)=1$.
	\[W = \left\{ \begin{bmatrix} a & b \\ c & d \end{bmatrix} \in M_{2\times2} \ \bigg| \ a + d = 1 \right\}\]
	Dengan operasi penjumlahan dan perkalian skalar standar matriks, apakah himpunan $W$ merupakan ruang vektor atau bukan
\end{enumerate}

Jawaban:\\
Sebuah ruang vektor harus memenuhi sepuluh aksioma ruang vektor yang diantaranya:
\begin{itemize}
	\item Tertutup pada penjumlahan
	\[u,v \in V:u+v \in V\]
	\item Komutatif pada penjumlahan
	\[u,v \in V:u+v = v+u\]
	\item Asosiatif pada penjumlahan
	\[u,v,w \in V:(u+v)+w = u+(v+w)\]
	\item Elemen identitas, penjumlahan dengan vektor nol tidak akan mengubah nilai
	\[0,v \in V:0+v = v+0 = v\]
	\item Tertutup pada perkalian skalar
	\[k \in \mathbb{R}, v \in V: kv \in V\]
	\item Distributif terhadap penjumlahan vektor
	\[k \in \mathbb{R}, u,v \in V: k(u+v) = ku + kv\]
	\item Distributif terhadap penjumlahan skalar
	\[k,l \in \mathbb{R}, u \in V: (k+l)u = ku + lu\]
	\item Asosiatif terhadap perkalian skalar
	\[k,l \in \mathbb{R}, u \in V: k(lv) = (kl)v\]
	\item Identitas skalar
	\[v \in V: 1v=v\]
	\item Invers penjumlahan
	\[v, -v \in V: v+(-v)=0\]
\end{itemize}
Jika ada salah satu aksioma yang tidak terpenuhi, maka himpunan yang diketahui bukan lagi ruang vektor.
\begin{enumerate}
	\item Himpunan $V$ bukanlah ruang vektor. Himpunan $V$ tidak memenuhi aksioma $1\mathbf{u}=\mathbf{u}$.\\
	Misalkan $\mathbf{u}=(u_1,u_2)$:
	\[1\mathbf{u}=\mathbf{u}\]
	\[1(u_1,u_2)=(u_1,u_2)\]
	\[(u_1,u_2)\neq(u_1,0)\]
	
	\item Himpunan $W$ bukanlah ruang vektor. Himpunan $W$ tidak memenuhi aksioma $kV\in\mathbb{R}$.\\
	Ambil sembarang $V$ dan $k$ dengan $k\neq1$
	\[
		A=\begin{bmatrix}
			1 & 0 \\ 0 & 0
		\end{bmatrix},
		k=2		
	\]
	\[
		kA=2\cdot\begin{bmatrix}
			2(1) & 2(0) \\ 2(0) & 2(0)
		\end{bmatrix}
	\]
	\[
		kA=\begin{bmatrix}
			2 & 0 \\ 0 & 0
		\end{bmatrix}		
	\]
	\[tr(kA)=2+0=2\]
	Himpunan $W$ hanya akan memiliki $tr(M)=1$ jika dan hanya jika $k=1$ sedangkan hal tersebut tidak memenuhi untuk $k\in\mathbb{R}$
\end{enumerate}