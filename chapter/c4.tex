% Pertemuan 4
\section*{Pertemuan 4}
\begin{enumerate}
	\item Tentukan minor dan kofaktor dari matriks A
	\[
		A=\begin{bmatrix}
			8 & 3 & 0 \\ -4 & -9 & 1 \\ -7 & 5 & 0
		\end{bmatrix}			
	\]
	\item Hitung determinan dari matriks B
	\[
		B=\begin{bmatrix}
			5 & 8 & -7 \\ -6 & 3 & 4 \\ 8 & 1 & 3
		\end{bmatrix}			
	\]
	\item Tentukan semua nilai $\lambda$ sehingga $det(A)=0$
	\[
		A=\begin{bmatrix}
			1 - \lambda & 2 & 1 \\ 2 & 4 - \lambda & 2 \\ 1 & -1 & 3 - \lambda
		\end{bmatrix}			
	\]
	\item Tentukan determinan dengan kofaktor
	\[
		R=\begin{bmatrix}
			2 & -7 & -3 \\ 1 & -8 & 0 \\ 6 & 8 & 0
		\end{bmatrix}			
	\]
\end{enumerate}

Jawab:
\begin{enumerate}
	\item 
	\begin{align*}
	M_{11}&=\begin{vmatrix} -9 & 1 \\ 5 & 0 \end{vmatrix} = (-9)(0) - (1)(5) = -5; & C_{11}&=(-1)^{1+1}(-5)=-5 \\
	M_{12}&=\begin{vmatrix} -4 & 1 \\ -7 & 0 \end{vmatrix} = (-4)(0) - (1)(-7) = 7; & C_{12}&=(-1)^{1+2}(7)=-7 \\
	M_{13}&=\begin{vmatrix} -4 & -9 \\ -7 & 5 \end{vmatrix} = (-4)(5) - (-9)(-7) = -20 - 63 = -83; & C_{13}&=(-1)^{1+3}(-83)=-83 \\
	M_{21}&=\begin{vmatrix} 3 & 0 \\ 5 & 0 \end{vmatrix} = (3)(0) - (0)(5) = 0; & C_{21}&=(-1)^{2+1}(0)=0 \\
	M_{22}&=\begin{vmatrix} 8 & 0 \\ -7 & 0 \end{vmatrix} = (8)(0) - (0)(-7) = 0; & C_{22}&=(-1)^{2+2}(0)=0 \\
	M_{23}&=\begin{vmatrix} 8 & 3 \\ -7 & 5 \end{vmatrix} = (8)(5) - (3)(-7) = 40 - (-21) = 61; & C_{23}&=(-1)^{2+3}(61)=-61 \\
	M_{31}&=\begin{vmatrix} 3 & 0 \\ -9 & 1 \end{vmatrix} = (3)(1) - (0)(-9) = 3; & C_{31}&=(-1)^{3+1}(3)=3 \\
	M_{32}&=\begin{vmatrix} 8 & 0 \\ -4 & 1 \end{vmatrix} = (8)(1) - (0)(-4) = 8; & C_{32}&=(-1)^{3+2}(8)=-8 \\
	M_{33}&=\begin{vmatrix} 8 & 3 \\ -4 & -9 \end{vmatrix} = (8)(-9) - (3)(-4) = -72 - (-12) = -60; & C_{33}&=(-1)^{3+3}(-60)=-60
	\end{align*}
	\item Metode Sarrus
	\[
		det(B)=
		\begin{array}{|ccc|cc|}
			5 & 8 & -7 & 5 & 8 \\ 
			-6 & 3 & 4 & -6 & 3 \\ 
			8 & 1 & 3 & 8 & 1
		\end{array}
	\]
	\[
		det(B)=[(5)(3)(3)+(8)(4)(8)+(-7)(-6)(1)]-[(-7)(3)(8)+(5)(4)(1)+(8)(-6)(3)]
	\]
	\[det(B)=(45 + 256 + 42) - (-168 + 20 - 144)\]
	\[det(B)=343-(-292)\]
	\[det(B)=635\]
	\item Metode Ekspansi Kofaktor
	\[
		(1-\lambda)\begin{vmatrix} 4-\lambda & 2 \\ -1 & 3-\lambda \end{vmatrix} -
		2\begin{vmatrix} 2 & 2 \\ 1 & 3-\lambda \end{vmatrix} + 
		1\begin{vmatrix} 2 & 4-\lambda \\ 1 & -1 \end{vmatrix}
	\]
	Bagian 1:
	\[(1-\lambda) [(4-\lambda)(3-\lambda) - (2)(-1)]\]
	\[= (1-\lambda) [(12 - 4\lambda - 3\lambda + \lambda^2) + 2]\]
	\[= (1-\lambda) [ \lambda^2 - 7\lambda + 14]\]
	Bagian 2:
	\[-2[(2)(3-\lambda) - (2)(1)]\]
	\[= -2[6 - 2\lambda - 2]\]
	\[= -2[4 - 2\lambda]\]
	\[= -8 + 4\lambda\]
	Bagian 3:
	\[1[ (2)(-1) - (4-\lambda)(1)]\]
	\[= 1 [ -2 - 4 + \lambda]\]
	\[= \lambda - 6\]
	Gabungkan ketiganya:
	\[(1-\lambda)(\lambda^2-7\lambda +14)+(4\lambda -8)+(\lambda -6)=0\]
	\[(1 - \lambda)(\lambda^2 - 7\lambda + 14) + (5\lambda - 14)=0\]
	\[(-\lambda^3 + 8\lambda^2 - 21\lambda + 14) + (5\lambda - 14)=0\]
	\[-\lambda^3 + 8\lambda^2 - 16\lambda=0\]
	Kalikan dengan -1:
	\[\lambda^3 - 8\lambda^2 + 16\lambda=0\]
	\[\lambda(\lambda^2 - 8\lambda + 16)=0\]
	\[\lambda(\lambda - 4)(\lambda - 4) = 0\]
	\[\lambda(\lambda - 4)^2 = 0\]
	$\therefore$ Sehingga didapat nilai-nilai eigennya adalah $\lambda_{1}=0,\lambda_{2}=4,\lambda_{3}=4$
	\item Metode Ekspansi Kofaktor terhadap Baris 1
	\[
		det(R)=
		2\begin{vmatrix} -8 & 0 \\ 8 & 0 \end{vmatrix} +
		7\begin{vmatrix} 1 & 0 \\ 6 & 0 \end{vmatrix} -
		3\begin{vmatrix} 1 & -8 \\ 6 & 8 \end{vmatrix}
	\]
	\[det(R)=(2)(0)+(7)(0)-(3)(56)\]
	\[det(R)=-168\]
\end{enumerate}