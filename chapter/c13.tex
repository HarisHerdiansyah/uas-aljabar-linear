% Pertemuan 13
\section*{Pertemuan 13}
\begin{enumerate}
	\item Diketahui $W$ adalah himpunan semua vektor di $\mathbb{R}^3$ yang memenuhi persamaan linear homogen $x-2y+3z=0$.
	\[W=\{(x,y,z)\in\mathbb{R}^3 \ | \ x-2y+3z=0\}\]
	Buktikan bahwa $W$ adalah subruang dari $\mathbf{R}^3$.
	
	\item Diketahui $D$ adalah himpunan semua matriks diagonal berukuran $2\times2$. Matriks diagonal adalah matriks di mana entri di luar diagonal utama bernilai nol.
	\[D = \left\{
	\begin{bmatrix} a & 0 \\ 0 & b \end{bmatrix} \ 
	\bigg| \ a, b \in \mathbb{R} 
	\right\}\]
	Buktikan bahwa $D$ adalah subruang dari ruang vektor $M_{2\times2}$.
	
	\item Diketahui $S$ adalah himpunan semua $P_2$ yang bernilai nol ketika $x=1$.
	\[S=\{p(x)=ax^2+bx+c \ | \ p(1)=0\}\]
	Buktikan bahwa $S$ adalah subruang dari $P_2$.
\end{enumerate}

\textbf{Jawaban:}
Subruang vektor tidak jauh berbeda dengan ruang vektor sebelumnya, hanya saja subruang vektor hanya perlu memenuhi empat aksioma saja.
\begin{enumerate}
	\item $W=\{(x,y,z)\in\mathbb{R}^3 \ | \ x-2y+3z=0\}$
	\begin{itemize}
		\item Akan dibuktikan jika $W\neq\{\}$
		Misal diambil vektor nol: $0 = (0,0,0)$. Substitusi ke persamaan
		\[x-2y+3z=0\]
		\[0-2(0)+3(0)=0\]
		\[0=0\]
		$\therefore W$ bukan himpunan kosong. 
		\item $\therefore W\in\mathbb{R}^3$
		\item Akan dibuktikan jika $W$ tertutup pada penjumlahan $(\mathbf{u}+\mathbf{v} \in W)$.
		Misal $\mathbf{u}=(x_1,y_1,z_1)$ dan $\mathbf{v}=(x_2,y_2,z_2)$.
		\[\mathbf{u}+\mathbf{v}=0\]
		\[(x_1+x_2)-2(y_1+y_2)+3(z_1+z_2)=0\]
		\[(x_1-2y_1+3z_1)+(x_2-2y_2+3z_2)=0\]
		\[0=0\]
		$\therefore W$ tertutup pada penjumlahan.
		\item Akan dibuktikan jika $W$ tertutup pada perkalian skalar.
		\[k\mathbf{u}=0\]
		\[k(x_1-2y_1+3z_1)=0\]
		\[k(0)=0\]
		\[0=0\]
		$\therefore W$ tertutup pada perkalian skalar.\\
		Sehingga $W$ adalah subruang di $\mathbb{R}^3$.
	\end{itemize}
	
	\item $D = \left\{
	\begin{bmatrix} a & 0 \\ 0 & b \end{bmatrix} \ 
	\bigg| \ a, b \in \mathbb{R} 
	\right\}$
	\begin{itemize}
		\item Akan dibuktikan jika $D\neq\{\}$
		Misal diambil matriks nol, di mana $a=0$ dan $b=0$, maka:
		\[\begin{bmatrix}
			0 & 0 \\ 0 & 0
		\end{bmatrix}\]
		$\therefore D$ bukan himpunan kosong dan nilai di luar diagonal tetap nol.
		\item $\therefore D \in M_{2\times2}$
		\item Akan dibuktikan jika $D$ tertutup pada penjumlahan.
		Misal diambil matriks $A$ dan $B$:
		\[
			A=\begin{bmatrix}
				a_1 & 0 \\ 0 & a_2	
			\end{bmatrix},
			B=\begin{bmatrix}
				b_1 & 0 \\ 0 & b_2
			\end{bmatrix}						
		\]
		\[
			A+B=\begin{bmatrix}
				a_1 + a_2 & 0 + 0 \\ 0 + 0 & b_1 + b_2
			\end{bmatrix}
		\]
		\[
			A+B=\begin{bmatrix}
				a_1 + a_2 & 0 \\ 0 & b_1 + b_2
			\end{bmatrix}					
		\]
		$\therefore D$ tertutup pada penjumlahan.
		\item Akan dibuktikan jika $D$ tertutup pada perkalian skalar.
		\[
			kA=k\begin{bmatrix}
				a_1 & 0 \\ 0 & b_1
			\end{bmatrix}
			=\begin{bmatrix}
				ka_1 & 0 \\ 0 & kb_1
			\end{bmatrix}						
		\]
		$\therefore D$ tertutup pada perkalian skalar.\\
		Sehingga $D$ adalah subruang di $M_{2\times2}$
	\end{itemize}
	
	\item $S=\{p(x)=ax^2+bx+c \ | \ p(1)=0\}$
	\begin{itemize}
		\item Uji dengan polinomial nol.
		\[0(x)=0x^2+0x+0=0\]
		$\therefore$ Dengan nilai $x=1$, polinomial akan tetap bernilai nol sehingga $S$ bukan himpunan kosong.
		\item $S$ adalah ruang vektor dengan derajat terbesar 2 sehingga $S \in P_2$.
		\item Akan dibuktikan jika $D$ tertutup pada penjumlahan.
		Misal diambil $p(x)$ dan $q(x)$ di mana $p(1)=0$ dan $q(1)=0$.
		\[r(x)=(p+q)(x)=p(x)+q(x)\].
		Periksa nilai di $x=1$:
		\[r(1)=p(1)+q(1)=0+0=0\]
		$\therefore S$ tertutup pada penjumlahan.
		\item Akan dibuktikan jika $S$ tertutp pada perkalian skalar.
		Misal $p(x) \in S$ dan $k \in \mathbb{R}$.
		\[m(x)=(kp)(x)=k\cdotp(x)\]
		Periksa nilai di $x=1$:
		\[m(1)=k\cdotp(1)=k\cdot0=0\]
		$\therefore S$ tertutup pada perkalian skalar.\\
		Sehingga $S$ adalah subruang $P_2$.
	\end{itemize}
\end{enumerate}