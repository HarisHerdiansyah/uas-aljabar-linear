% Pertemuan 9
\section*{Pertemuan 9}
\begin{enumerate}
	\item Tentukan basis ruang null dan basis ruang baris dari matriks A
	\[
		A=\begin{bmatrix}
			1 & -2 & 1 & 3 \\ 2 & -4 & 3 & 7 \\ -1 & 2 & 0 & -2
		\end{bmatrix}			
	\]
	\item Tentukan basis ruang baris dan basis ruang kolom melalui inspeksi
	\[
		R=\begin{bmatrix}
			1 & 5 & 0 & 2 & -1 \\
			0 & 0 & 1 & -3 & 4 \\
			0 & 0 & 0 & 1 & 7 \\
			0 & 0 & 0 & 0 & 0
		\end{bmatrix}			
	\]
	\item Tentukan rank dan nullity dari matriks B
	\[
		B=\begin{bmatrix}
			1 & 2 & -1 & 4 \\ 2 & 4 & -2 & 8 \\ 1 & 3 & 2 & 0 \\ -1 & -1 & 4 & -8
		\end{bmatrix}			
	\]
\end{enumerate}

Jawaban:
\begin{enumerate}
	\item Lakukan operasi baris elementer hingga didapat bentuk matriks eselon baris:
	\[
		\begin{bmatrix}
			1 & -2 & 1 & 3 \\ 2 & -4 & 3 & 7 \\ -1 & 2 & 0 & -2
		\end{bmatrix}
		\xrightarrow{\substack{-2B1+B2 \\ B1 + B3}}
		\begin{bmatrix}
			1 & -2 & 1 & 3 \\ 0 & 0 & 1 & 1 \\ 0 & 0 & 1 & 1
		\end{bmatrix}
		\xrightarrow{-B2+B3}
		\begin{bmatrix}
			1 & -2 & 1 & 3 \\ 0 & 0 & 1 & 1 \\ 0 & 0 & 0 & 0
		\end{bmatrix}
	\]
	Ambil baris tidak nol sebagai basis ruang baris:
	\[Row(A)=\{(1,-2,1,3),(0,0,1,1)\}\]
	Menentukan basis ruang null:
	\[x_3 + x_4 = 0 \longrightarrow x_3 = -x_4\]
	\[x_1 - 2x_2 + x_3 + 3x_4 = 0\]
	\[x_1 = 2x_2 - x_3 - 3x_4\]
	Misal $x_2 = s$ dan $x_4 = t$:
	\[x_3 = -t\]
	\[x_1 = 2s - (-t) - 3t\]
	\[x_1 = 2s - 2t\]
	Vektor solusi:
	\[
		\begin{bmatrix}
			x_1 \\ x_2 \\ x_3 \\ x_4
		\end{bmatrix}=
		\begin{bmatrix}
			2s - 2t \\ s \\ -t \\ t		
		\end{bmatrix}=
		s\begin{bmatrix}
			2 \\ 1 \\ 0 \\ 0
		\end{bmatrix}+
		t\begin{bmatrix}
			-2 \\ 0 \\ -1 \\ 1
		\end{bmatrix}
	\]
	\[
		Null(A)=
		\left\{
		\begin{bmatrix}
			2 \\ 1 \\ 0 \\ 0
		\end{bmatrix},
		\begin{bmatrix}
			-2 \\ 0 \\ -1 \\ 1
		\end{bmatrix}
		\right\}	
	\]
	
	\item Basis ruang baris diambil dari baris yang membuat pivot dan basis ruang kolom diambil dari kolom yang memuat pivot.
	\[Row(B)=\{(1,5,0,2,-1),(0,0,1,-3,4),(0,0,0,1,7)\}\]
	\[
		Col(B)=\left\{
		\begin{bmatrix}
			1 \\ 0 \\ 0 \\ 0
		\end{bmatrix},
		\begin{bmatrix}
			0 \\ 1 \\ 0 \\ 0
		\end{bmatrix},
		\begin{bmatrix}
			2 \\ -3 \\ 1 \\ 0
		\end{bmatrix}
		\right\}
	\]
	
	\item Lakukan operasi baris elementer
	\[
		\begin{bmatrix}
			1 & 2 & -1 & 4 \\
			2 & 4 & -2 & 8 \\
			1 & 3 & 2 & 0 \\
			-1 & -1 & 4 & -8
		\end{bmatrix}	
		\xrightarrow{\substack{-2B1+B2 \\ -B1+B3 \\ B1+B4}}
		\begin{bmatrix}
			1 & 2 & -1 & 4 \\
			0 & 0 & 0 & 0 \\
			0 & 1 & 3 & -4 \\
			0 & 1 & 3 & -4
		\end{bmatrix}
		\xrightarrow{\text{B2}\leftrightarrow\text{B3}}	
		\begin{bmatrix}
			1 & 2 & -1 & 4 \\
			0 & 1 & 3 & -4 \\
			0 & 0 & 0 & 0 \\
			0 & 1 & 3 & -4
		\end{bmatrix}			
	\]
	\[
		\xrightarrow{-B2+B4}
		\begin{bmatrix}
			1 & 2 & -1 & 4 \\
			0 & 1 & 3 & 4 \\
			0 & 0 & 0 & 0 \\
			0 & 0 & 0 & 0
		\end{bmatrix}			
	\]
	Jumlah baris tidak nol:
	\[Rank(B)=2\]
	Menentukan nullity:
	\[\text{Total kolom }(n) = 4\]
	\[Nullity = Rank(B) - n = 4 - 2 = 2\]
\end{enumerate}