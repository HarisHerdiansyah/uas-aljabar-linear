% Pertemuan 2
\section*{Pertemuan 2}
Diketahui matriks $A, B, C, D, $ dan $E$ sebagai berikut: \\
\[
	A = \begin{bmatrix}
		2 & 1 \\
		-4 & 3 \\
		5 & 0
	\end{bmatrix},
	\quad
	B = \begin{bmatrix}
		-1 & 5 \\
		2 & 4
	\end{bmatrix},
	\quad
	C = \begin{bmatrix}
		4 & -2 & 0 \\
		1 & 7 & 3
	\end{bmatrix}
\]
\[
	D = \begin{bmatrix}
		7 & 1 & 3 \\
		-2 & 0 & 4 \\
		1 & 6 & 9
	\end{bmatrix},
	\quad
	E = \begin{bmatrix}
		-3 & 0 & 8 \\
		2 & 5 & 1 \\
		-1 & 4 & 6
	\end{bmatrix}
\]

Tentukan hasil dari ekspresi berikut jika matriks terdefinsi:
\begin{enumerate}
	\item $B^{T}CC^{T}-A^{T}A$
	\item $D^{T}E^{T}-(ED)^{T}$
\end{enumerate}

\textbf{Jawaban:} \\
Untuk memudahkan perhitungan selanjutnya, cari terlebih dahulu transpose dari setiap matriks. \textbf{Transpose matriks} adalah bentuk matriks di mana posisi baris ditukar posisi ke bawah menjadi kolom, begitu juga untuk posisi kolom ditukar posisi ke samping menjadi baris, hal ini juga akan mengubah ordo matriks jika bukan matriks persegi.

\[
	A^{T} = \begin{bmatrix}
		2 & -4 & 5 \\ 1 & 3 & 0
	\end{bmatrix},
	\quad
	B^{T} = \begin{bmatrix}
		-1 & 2 \\ 5 & 4
	\end{bmatrix},
	\quad
	C^{T} = \begin{bmatrix}
		4 & 1 \\ -2 & 7 \\ 0 & 3
	\end{bmatrix}
\]
\[
	D^{T} = \begin{bmatrix}
		7 & -2 & 1 \\ 1 & 0 & 6 \\ 3 & 4 & 9 
	\end{bmatrix},
	\quad
	E^{T} = \begin{bmatrix}
		-3 & 2 & -1 \\ 0 & 5 & 4 \\ 8 & 1 & 6
	\end{bmatrix}
\]

\begin{enumerate}
	\item Hitung $A^{T}A$:
	\[A^T A = \begin{bmatrix} 2 & -4 & 5 \\ 1 & 3 & 0 \end{bmatrix} \begin{bmatrix} 2 & 1 \\ -4 & 3 \\ 5 & 0 \end{bmatrix} = \begin{bmatrix} (4+16+25) & (2-12+0) \\ (2-12+0) & (1+9+0) \end{bmatrix} = \begin{bmatrix} 45 & -10 \\ -10 & 10 \end{bmatrix}\]
	
	Hitung $CC^T$:	
	\[C C^T = \begin{bmatrix} 4 & -2 & 0 \\ 1 & 7 & 3 \end{bmatrix} \begin{bmatrix} 4 & 1 \\ -2 & 7 \\ 0 & 3 \end{bmatrix} = \begin{bmatrix} (16+4+0) & (4-14+0) \\ (4-14+0) & (1+49+9) \end{bmatrix} = \begin{bmatrix} 20 & -10 \\ -10 & 59 \end{bmatrix}\]
	
	Hitung $B(CC^T)$:
	\[B^T (C C^T) = \begin{bmatrix} -1 & 2 \\ 5 & 4 \end{bmatrix} \begin{bmatrix} 20 & -10 \\ -10 & 59 \end{bmatrix} = \begin{bmatrix} (-20-20) & (10+118) \\ (100-40) & (-50+236) \end{bmatrix} = \begin{bmatrix} -40 & 128 \\ 60 & 186 \end{bmatrix}\]
	
	Hitung hasil akhir:
	\[\begin{bmatrix} -40 & 128 \\ 60 & 186 \end{bmatrix} - \begin{bmatrix} 45 & -10 \\ -10 & 10 \end{bmatrix} = \begin{bmatrix} -85 & 138 \\ 70 & 176 \end{bmatrix}\]
	
	Dapat dilihat bahwa perkalian matriks menghasil ordo baru yang bernilai baris matriks pertama $\times$ kolom matriks kedua.
	
	\item
	Hitung dulu $D^{T}E^{T}$:
	\[
		D^{T}E^{T} = \begin{bmatrix}
			7 & -2 & 1 \\ 1 & 0 & 6 \\ 3 & 4 & 9 
		\end{bmatrix}
		\cdot
		\begin{bmatrix}
			-3 & 2 & -1 \\ 0 & 5 & 4 \\ 8 & 1 & 6
		\end{bmatrix}
	\]
	\[
		D^{T}E^{T} = \begin{bmatrix}
			(7)(-3)+(-2)(0)+(1)(8) 
				& (7)(2)+(-2)(5)+(1)(1) 
				& (7)(-1)+(-2)(4)+(1)(6) \\
			(1)(-3)+(0)(0)+(6)(8) 
				& (1)(2)+(0)(5)+(6)(1) 
				& (1)(-1)+(0)(4)+(6)(6) \\
			(3)(-3)+(4)(0)+(9)(8) 
				& (3)(2)+(4)(5)+(9)(1) 
				& (3)(-1)+(4)(4)+(9)(6) \\
		\end{bmatrix}
	\]
	\[
		D^{T}E^{T} = \begin{bmatrix}
			-13 & 5 & -9 \\ 45 & 8 & 35 \\ 63 & 35 & 67
		\end{bmatrix}
	\]
	
	Hitung $ED$:
	\[
		ED = \begin{bmatrix}
			-3 & 0 & 8 \\ 2 & 5 & 1 \\ -1 & 4 & 6
		\end{bmatrix}
		\cdot
		\begin{bmatrix}
			7 & 1 & 3 \\ -2 & 0 & 4 \\ 1 & 6 & 9
		\end{bmatrix}
	\]
	\[
		ED = \begin{bmatrix}
			(-3)(7) + (0)(-2) + (8)(1)
				& (-3)(1) + (0)(0) + (8)(6)
				& (-3)(3) + (0)(4) + (8)(9) \\
			(2)(7) + (5)(-2) + (1)(1)
				& (2)(1) + (5)(0) + (1)(6)
				& (2)(3) + (5)(4) + (1)(9) \\
			(-1)(7) + (4)(-2) + (6)(1)
				& (-1)(1) + (4)(0) + (6)(6)
				& (-1)(3) + (4)(4) + (6)(9) \\
		\end{bmatrix}			
	\]
	\[
		ED = \begin{bmatrix}
			-13 & 45 & 63 \\ 5 & 8 & 35 \\ -9 & 35 & 67
		\end{bmatrix}			
	\]
	
	Hitung hasil akhir:
	\[
		D^{T}E^{T}-(ED)^{T} = \begin{bmatrix}
			-13 & 5 & -9 \\ 45 & 8 & 35 \\ 63 & 35 & 67
		\end{bmatrix}
		-
		\begin{bmatrix}
			-13 & 5 & -9 \\ 45 & 8 & 35 \\ 63 & 35 & 67
		\end{bmatrix}
		=
		\begin{bmatrix}
			0 & 0 & 0 \\ 0 & 0 & 0 \\ 0 & 0 & 0
		\end{bmatrix}
	\]
\end{enumerate}