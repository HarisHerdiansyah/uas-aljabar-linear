% Pertemuan 5
\section*{Pertemuan 5}
\begin{enumerate}
	\item Selesaikan sistem persamaan linear berikut
	\[
		\begin{cases}
			x_{1} + 2x_{2} - x_{3} + 3x_{4} = 4 \\
			2x_{1} + 4x_{2} - 2x_{3} + 7x_{4} = 10 \\
			-x_{1} - 2x_{2} + x_{3} - 4x_{4} = -6
		\end{cases}	
	\]	
	
	\item Tentukan nilai konstanta $k$ agar sistem persamaan berikut:
	\[
		\begin{cases}
			x + y + kz = 1 \\ x + ky + z = 1 \\ kx + y + z = 1
		\end{cases}
	\]
	\begin{enumerate}
		\renewcommand{\labelenumi}{\alph{enumi}.}
		\item Memiliki tepat satu solusi.
		\item Tidak memiliki solusi.
		\item Memiliki solusi tak hingga.
	\end{enumerate}
\end{enumerate}

Jawab:
\begin{enumerate}
	\item Bentuk augmented matrix:
	\[
		\left[
		\begin{array}{cccc|c}
			1 & 2 & -1 & 3 & 4 \\
			2 & 4 & -2 & 7 & 10 \\
			-1 & -2 & 1 & -4 & -6
		\end{array}
		\right]
	\]
	Lakukan operasi baris elementer:
	\[
		\left[
		\begin{array}{cccc|c}
			1 & 2 & -1 & 3 & 4 \\
			2 & 4 & -2 & 7 & 10 \\
			-1 & -2 & 1 & -4 & -6
		\end{array}
		\right]
		\xrightarrow{-2B1+B2}
		\left[
		\begin{array}{cccc|c}
			1 & 2 & -1 & 3 & 4 \\
			0 & 0 & 0 & 1 & 2 \\
			-1 & -2 & 1 & -4 & -6
		\end{array}
		\right]
		\xrightarrow{B1+B3}
		\left[
		\begin{array}{cccc|c}
			1 & 2 & -1 & 3 & 4 \\
			0 & 0 & 0 & 1 & 2 \\
			0 & 0 & 0 & -1 & -2
		\end{array}
		\right]
	\]
	\[
		\xrightarrow{-3B2+B1}
		\left[
		\begin{array}{cccc|c}
			1 & 2 & -1 & 0 & -2 \\
			0 & 0 & 0 & 1 & 2 \\
			0 & 0 & 0 & -1 & -2
		\end{array}
		\right]
		\xrightarrow{B2+B3}
		\left[
		\begin{array}{cccc|c}
			1 & 2 & -1 & 0 & -2 \\
			0 & 0 & 0 & 1 & 2 \\
			0 & 0 & 0 & 0 & 0
		\end{array}
		\right]
	\]
	Interpretasi solusi:
	\[x_{4} = 2\]
	\[x_{1} + 2x_{2} - x_{3} = -2\]
	\[x_{1} = -2 - 2x_{2} + x_{3}\]
	Misal $x_{2} = s$ dan $x_{3} = t$, maka:
	\[x_{1} = -2 - 2s + t\]
	Solusi persamaan:
	\[
		\begin{bmatrix}
			x_{1} \\ x_{2} \\ x_{3} \\ x_{4}
		\end{bmatrix}
		=
		\begin{bmatrix}
			-2 - 2s -+ t \\ s \\ t \\ 2
		\end{bmatrix}	
	\]
	\[
		\begin{bmatrix}
			x_{1} \\ x_{2} \\ x_{3} \\ x_{4}
		\end{bmatrix}
		=
		\begin{bmatrix}
			-2 \\ 0 \\ 0 \\ 2
		\end{bmatrix} +
		s\begin{bmatrix}
			-2 \\ 1 \\ 0 \\ 0
		\end{bmatrix} +
		t\begin{bmatrix}
			1 \\ 0 \\ 1 \\ 0
		\end{bmatrix}
	\]
	
	\item Penyelesaian dilakukan dengan analisis determinan. Jika $det(A)\neq0$, maka didapat solusi tunggal. Jika $det(A)=0$, maka perlu diperiksa lebih lanjut apakah terdapat tak hingga solusi atau tidak ada solusi.
	
	Bentuk matriks koefisien:
	\[
		A = \begin{bmatrix}
			1 & 1 & k \\ 1 & k & 1 \\ k & 1 & 1
		\end{bmatrix}			
	\]
	Hitung determinan:
	\[det(A)=1(k-1)-1(1-k)+k(1-k^2)\]
	\[=k-1-1+k+k+k^3\]
	\[=-k^3+3k-2\]
	Kalikan dengan $-1$
	\[k^3-3k+2\]
	Cari pembuat $0$ dengan mencari akar persamaan, pencarian akar menggunakan Metode Horner dan perlu diketahui salah satu bilangan pembuat nol dengan mengambil asumsi dari faktor konstanta (pada kasus ini adalah $2$), yaitu $\{1,2\}$.\\
	Ambil $1$ terlebih dahulu.
	\[(1)^3-3(1)+2=1-3+2=0 \hspace{1cm} \text{\emph{1 adalah akar}} \]
	Ambil koefisien persamaan, yaitu $\{1,0,-3,2\}$:
	\[
		\begin{array}{c|ccccc}
		1 & 1 & 0 & -3 & 2 & \\
		& & 1 & 1 & -2 & + \\
		\hline
		& 1 & 1 & -2 & 0
		\end{array}
	\]
	Didapatkan koefiesien baru $\{1, 1, -2\}$, buat persamaan baru:
	\[k^2+k-2=0\]
	Lanjutkan pencarian pembuat nol determinan:
	\[(k-1)(k^2+k-2)=0\]
	\[(k-1)(k+2)(k-1)=0\]
	Maka didapat akar-akar persamaan (pembuat nol), yaitu $k=1$ atau $k=-2$.
	
	Analisis kasus
	\begin{enumerate}
		\renewcommand{\labelenumi}{\alph{enumi}.}
		\item Persamaan memiliki solusi tunggal/unik jika $det(A)\neq0$ maka nilai $k$ yang salah adalah $k\neq1$ dan $k\neq-2$
		\item Uji salah satu akar, misal $k=1$
		\[
			\begin{cases}
				x + y + z = 1 \\ x + y + z = 1 \\ x + y + z = 1
			\end{cases}		
		\]
		Semua persamaan identik dan terdapat dua variabel bebas, maka SPL memiliki tak hingga solusi untuk $k=1$
		
		\item Uji $k=-2$
		Buat augmented matrix
		\[
			\left[
			\begin{array}{ccc|c}
				1 & 1 & -2 & 1 \\ 1 & -2 & 1 & 1 \\ -2 & 1 & 1 & 1
			\end{array}
			\right]				
		\]
		Lakukan operasi baris elementer hingga didapat bentuk matriks eselon baris tereduksi:
		\[
			\left[
			\begin{array}{ccc|c}
				1 & 0 & -1 & 1 \\ 0 & 1 & -1 & 0 \\ 0 & 0 & 0 & 3
			\end{array}
			\right]
		\]
		Terdapat inkonsisten pada persamaan, sehingga SPL tidak memiliki solusi ketika $k=-2$
	\end{enumerate}
\end{enumerate}