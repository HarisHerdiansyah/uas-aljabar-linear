% Pertemuan 14
\section*{Pertemuan 14}
\begin{enumerate}
	\item Diketahui himpunan vektor $B=\{v_1,v_2,v_3\}$ di ruang $\mathbb{R}^3$ dengan:
	\[v_1=(1,0,1), \ v_2=(0,1,1) \ v_3=(1,1,0)\]
	\begin{enumerate}
		\renewcommand{\labelenumi}{\alph{enumi}.}
		\item Tunjukkan bahwa himpunan $B$ adalah basis untuk $\mathbb{R}^3$.
		\item Tentukan koordinat vektor $\mathbf{u}=(3,5,2)$ relatif terhadap basis $B$ atau $(\mathbf{u})_B.$
	\end{enumerate}
	
	\item Diketahui himpunan vektor $S=\{u_1,u_2,u_3\}$ sebagai berikut:
	\[u_1=(1,-2,3), \ v_2=(2,1,-1) \ v_3=(3,-1,2)\]
	Periksa apakah himpunan $S$ bebas linear atau bergantung linear
\end{enumerate}

\textbf{Jawaban:}
\begin{enumerate}
	\item Basis adalah kondisi di mana sebuah vektor dalam sebuah himpunan memenuhi dua syarat, yaitu membangun dan bebas linear. Adapun secara teorema, matriks-matriks yang bebas linear, memiliki determinan $\neq$ 0 sehingga dapat dikatakan ketika $det(A)\neq0$ merupakan sebuah basis.
	\begin{enumerate}
		\renewcommand{\labelenumi}{\alph{enumi}.}
		\item $B$ adalah basis di $\mathbb{R}^3$ jika determinan $\neq$ 0
		\[B=\begin{bmatrix}
			1 & 0 & 1 \\ 0 & 1 & 1 \\ 1 & 1 & 0
		\end{bmatrix}\]
		\[
			det(A)=1\begin{vmatrix}
				1 & 1 \\ 1 & 0
			\end{vmatrix}
			-0+1\begin{vmatrix}
				0 & 1 \\ 1 & 1
			\end{vmatrix}							
		\]
		\[=1(0-1)+1(0-1)\]
		\[=-1-1=-2\]
		$\therefore B$ merupakan basis untuk $\mathbb{R}^3$ dengan $det(A)=-2\neq0$.
		\item Koordinat $(\mathbf{u}_B)$
		\[
			\begin{bmatrix}
				1 & 0 & 1 \\ 0 & 1 & 1 \\ 1 & 1 & 0
			\end{bmatrix}
			\begin{bmatrix}
				k_1 \\ k_2 \\ k_3
			\end{bmatrix}=
			\begin{bmatrix}
				3 \\ 5 \\ 2
			\end{bmatrix}						
		\]
		\[
			\begin{bmatrix}
				k_1 + k_3 \\ k_2 + k_3 \\ k_1 + k_2
			\end{bmatrix}=
			\begin{bmatrix}
				3 \\ 5 \\ 2
			\end{bmatrix}						
		\]
		Didapat sistem persamaan linear sebagai berikut:
		\[k_1+k_3=3\]
		\[k_2+k_3=5\]
		\[k_1+k_2=2\]
		Lakukan manipulasi aljabar:
		\[k_2+k_3=5 \Rightarrow k_3=5-k_2\]
		\[k_1+5-k_2=3 \Rightarrow k_1-k_2=-2 \Rightarrow k_1=k_2-2\]
		\[k_2-2+k_2=2 \Rightarrow 2k_2=4 \Rightarrow k_2=2\]
		Substitusi $k_2=2$:
		\[k_3=5-2 \Rightarrow k_3=3\]
		\[k_1=2-2 \Rightarrow k_1=0\]
		$\therefore$ Sehingga koordinat $(\mathbf{u}_B)=(0,2,3)$
	\end{enumerate}
	
	\item Suatu matriks dikatakan bebas linear jika memuat solusi trivial atau solusi unik. Adapun, matriks yang bergantung linear adalah ketika matriks memuat solusi non-trivial di mana pivotnya adalah 0 sehingga nilai determinan menjadi 0. Maka dari itu, dapat dikatakan jika $det(A)=0$ matriks tersebut bergantung linear.\\
	Susun matriks $M$ lalu uji determinan:
	\[
		M=\begin{bmatrix}
			1 & 2 & 3 \\ - 2 & 1 & -1 \\ 3 & -1 & 2
		\end{bmatrix}
	\]
	\[\text{det}(M) = 1 \begin{vmatrix} 1 & -1 \\ -1 & 2 \end{vmatrix} - 2 \begin{vmatrix} -2 & -1 \\ 3 & 2 \end{vmatrix} + 3 \begin{vmatrix} -2 & 1 \\ 3 & -1 \end{vmatrix}\]
	\[= 1(2 - 1) - 2(-4 - (-3)) + 3(2 - 3)\]
	\[= 1(1) - 2(-1) + 3(-1)\]
	\[= 1 + 2 - 3 = 0\]
	$\therefore$ Karena $det(M)=0$ maka vektor-vektor yang diketahui bergantung linear.
\end{enumerate}