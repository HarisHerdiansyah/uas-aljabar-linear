% Pertemuan 10
\section*{Pertemuan 10}
\begin{enumerate}
	\item Diketahui matriks A
	\[A=\begin{bmatrix}4&0&1\\-2&1&0\\-2&0&1\end{bmatrix}\]
	Tentukan:
	\begin{enumerate}
		\renewcommand{\labelenumi}{\alph{enumi}.}
		\item Persamaan karakteristik dari matriks $A$.
		\item Nilai-nilai eigen dari matriks $A$.
		\item Basis ruang eigen yang bersesuaian dengan setiap nilai eigen.
	\end{enumerate}
	
	\item Diketahui matriks B
	\[B=\begin{bmatrix}2&0&0\\0&3&1\\0&1&3\end{bmatrix}\]
	Tentukan:
	\begin{enumerate}
		\renewcommand{\labelenumi}{\alph{enumi}.}
		\item Tentukan matriks $P$ yang mendiagonalisasi $B$.
		\item Tentukan matriks diagonal $D$.
	\end{enumerate}
\end{enumerate}

Jawaban:
\begin{enumerate}
	\item Ruang eigen adalah ruang vektor hasil transformasi yang melibatkan nilai-nilai eigen, di mana hasil transformasi hanya berubah secara skalar. Persamaan karakteristik adalah persamaan muncul melalui persamaan transformasi linear:
	\[Av = \lambda{v}\]
	\[Av - \lambda{v} = 0\]
	\[A(v - I\lambda) = 0\]
	Dapat dilihat, nilai $\lambda$ dikalikan dengan $I$ atau matriks identitas karena pada dasarnya lambda adalah skalar. Persamaan karakteristik adalah determinan dari $A(v - I\lambda)$ di mana determinan harus bernilai nol.\\
	Nilai eigen adalah nilai-nilai yang memenuhi persamaan untuk membangun basis eigen.\\
	Basis ruang eigen adalah vektor yang bersesuaian dengan setiap nilai eigen untuk membangun ruang eigen.
	\begin{enumerate}
		\renewcommand{\labelenumi}{\alph{enumi}.}
		\item Mencari persamaan karakteristik:
		\[det\left(
			\begin{bmatrix}
				\lambda-4 & 0 & -1 \\ 2 & \lambda-1 & 0 \\ 2 & 0 & \lambda-1
			\end{bmatrix}
		\right)=0\]
		Lakukan ekspansi kofaktor sepanjang kolom ke-2:
		\[(\lambda-1) \cdot \det \begin{bmatrix} \lambda-4 & -1 \\ 2 & \lambda-1 \end{bmatrix} = 0\]
		\[(\lambda-1) [(\lambda-4)(\lambda-1) - (-1)(2)] = 0\]
		\[(\lambda-1) [\lambda^2 - 5\lambda + 4 + 2] = 0\]
		\[(\lambda-1) [\lambda^2 - 5\lambda + 6] = 0\]
		\[(\lambda-1)(\lambda-2)(\lambda-3) = 0\]
		
		\item Berdasarkan hasil pemfaktoran, nilai-nilai eigennya adalah:
		\[\lambda_1=1,\lambda_2=2,\lambda_3=3\]
		
		\item Basis ruang eigen
	\end{enumerate}
	
	\item Dalam konteks ruang eigen, matriks $P$ adalah matriks yang dibangun dari basis-basis ruang eigen sehingga memenuhi $A=PDP^{-1}$. Adapun matriks $D$ sendiri adalah matriks diagonal yang setiap nilai diagonalnya diambil dari nilai-nilai eigen.
	\begin{enumerate}
		\renewcommand{\labelenumi}{\alph{enumi}.}
		\item Matriks $P$ dibangun dari vektor-vektor eigen sehingga cari terlebih dahulu vektor-vektor eigennya.
		\[
			I\lambda-B=\begin{bmatrix}
				\lambda-2 & 0 & 0 \\ 0 & \lambda-3 & -1 \\ 0 & -1 & \lambda-3
			\end{bmatrix}					
		\]
		Cari determinan $det(I\lambda-B):$
		\[
			det(I\lambda-B)=(\lambda-2)\cdot
			\begin{vmatrix}
				\lambda-3 & 1 \\ 1 & \lambda-3			
			\end{vmatrix}					
		\]
		\[det(I\lambda-B)=(\lambda-2)((\lambda-3)(\lambda-3)-1)\]
		\[det(I\lambda-B)=(\lambda-2)(\lambda^2-6\lambda+9-1)\]
		\[det(I\lambda-B)=(\lambda-2)(\lambda^2-6\lambda+8)\]
		\[det(I\lambda-B)=(\lambda-2)(\lambda-2)(\lambda-4)\]
		Didapat nilai eigennya adalah $\lambda_1=2,\lambda_2=2,\lambda_3=4$. Tentukan vektor-vektor eigennya:
		\begin{itemize}
			\item $\lambda=2$
			\[
				\begin{bmatrix}
					0&0&0\\0&-1&-1\\0&-1&-1
				\end{bmatrix}
				\begin{bmatrix}
					x\\y\\z
				\end{bmatrix}=
				\begin{bmatrix}
					0\\-y-z\\-y-z
				\end{bmatrix}											
			\]
			Misal $x=s, z=t$:
			\[-y-t=0\]\[y=-t\]
			Vektor eigen untuk $\lambda=2$:
			\[
				\begin{bmatrix}
					x\\y\\z
				\end{bmatrix}=
				\begin{bmatrix}
					s\\-t\\t
				\end{bmatrix}=
				s\begin{bmatrix}
					1\\0\\0
				\end{bmatrix}+
				t\begin{bmatrix}
					0\\-1\\1
				\end{bmatrix}				
			\]
			
			\item $\lambda=4$
			\[
				\begin{bmatrix}
					2&0&0\\0&1&-1\\0&-1&1
				\end{bmatrix}			
				\begin{bmatrix}
					x\\y\\z
				\end{bmatrix}=
				\begin{bmatrix}
					2x\\y-z\\-y+z
				\end{bmatrix}										
			\]
			\[2x=0 \longrightarrow x=0 \]
			\[y-z=0 \longrightarrow y=z \]
			Misal $z=t$, maka $y=t$. Vektor eigen untuk $\lambda=4$:
			\[
				\begin{bmatrix}
					x\\y\\z
				\end{bmatrix}=
				\begin{bmatrix}
					0\\t\\t
				\end{bmatrix}=
				t\begin{bmatrix}
					0\\1\\1
				\end{bmatrix}					
			\]
			Sehinggga matriks $P$ adalah:
			\[
				\begin{bmatrix}
					1&0&0\\0&-1&1\\0&1&1
				\end{bmatrix}							
			\]
		\end{itemize}
		
		\item Matriks diagonal $D$ adalah matriks yang entri diagonalnya diambil dari nilai-nilai eigen ($\lambda_1=2,\lambda_2=2,\lambda_3=4$):
		\[
			D=\begin{bmatrix}
				2&0&0\\0&2&0\\0&0&4
			\end{bmatrix}					
		\]
	\end{enumerate}
\end{enumerate}