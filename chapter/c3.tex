% Pertemuan 3
\section*{Pertemuan 3}
Tentukan apakah matriks-matriks berikut adalah:
\begin{itemize}
	\item Matriks Eselon Baris
	\item Matriks Eselon Baris Tereduksi
	\item Keduanya
	\item atau Bukan Keduanya
\end{itemize}


\begin{enumerate}
	\item 
	\[
		\begin{bmatrix}
			1 & 0 & 0 & 0 & 2 \\
			0 & 1 & 0 & 0 & 5 \\
			0 & 0 & 1 & 0 & 4 \\
			0 & 0 & 0 & 1 & 0		
		\end{bmatrix} 	
	\]
	
	\item 
	\[
		\begin{bmatrix}
			0 \\
			0 \\
			0 \\
			0	
		\end{bmatrix} 	
	\]
	
	\item 
	\[
		\begin{bmatrix}
			1 & -2 & 5 & 4 \\
			0 & 1 & 9 & 6 \\
			0 & 0 & 0 & 0		
		\end{bmatrix} 	
	\]
\end{enumerate}

\textbf{Jawaban:}\\
Matriks eselon baris dan matriks eselon baris tereduksi adalah bentuk matris yang telah dimodifikasi dengan operasi baris elementer (perkalian skalar, penjumlahan terhadap kelipatan, dan penukaran baris). Matriks eselon baris adalah matriks yang menghasilkan matriks segitiga atas dengan nilai di titik diagonal adalah 1 yang selanjut disebut 1 utama. Matriks eselon baris tereduksi adalah matriks diagonal yang setiap nilai diagonalnya adalah 1.
\begin{enumerate}
	\item Matriks eselon baris tereduksi karena terbentuk matriks diagonal dengan nilai setiap diagonalnya adalah 1
	\item Keduanya karena matriks nol tidak melanggar syarat matriks eselon baris maupun matriks eselon baris tereduksi. Pada matriks eselon baris, baris nol harus ditempatkan paling bawah dan semua baris nol sudah pada tempat yang sesuai dan tidak ada 1 utama yang harus ditukar baris.
	\item Matriks eselon baris karena nilai di atas 1 utama baris kedua tidak sama dengan nol, untuk menjadi matriks eselon baris tereduksi, nilai di atas dan di bawah 1 utama haruslah nol.
\end{enumerate}