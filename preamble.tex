% Packages untuk bahasa Indonesia
%\usepackage[indonesian]{babel}
\usepackage[utf8]{inputenc}
\usepackage[T1]{fontenc}

% Packages matematika
\usepackage{amsmath}
\usepackage{amssymb}
\usepackage{amsthm}
\usepackage{mathtools}

% Packages untuk gambar dan grafik
\usepackage{graphicx}
\usepackage{float}
\usepackage{tikz}
\usepackage{pgfplots}
\pgfplotsset{compat=1.18}

% Packages untuk layout
\usepackage{geometry}
\geometry{
    left=2cm, right=2cm, top=2cm, bottom=2cm
}
\usepackage{setspace}
\onehalfspacing

% Packages untuk hyperlink dan referensi
\usepackage{hyperref}
\hypersetup{
    colorlinks=true,
    linkcolor=blue,
    citecolor=blue,
    urlcolor=blue
}

% Packages untuk kode (jika diperlukan)
\usepackage{listings}
\usepackage{xcolor}

% Theorem environments
\newtheorem{theorem}{Teorema}[chapter]
\newtheorem{definition}{Definisi}[chapter]
\newtheorem{example}{Contoh}[chapter]
\newtheorem{solution}{Pembahasan}[chapter]

% Custom commands
\newcommand{\R}{\mathbb{R}}
\newcommand{\C}{\mathbb{C}}
\newcommand{\Z}{\mathbb{Z}}